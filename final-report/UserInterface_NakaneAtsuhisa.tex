\documentclass[dvipdfmx]{jsarticle}
% Language setting
% Replace `english' with e.g. `spanish' to change the document language
\usepackage[english]{babel}

% Set page size and margins
% Replace `letterpaper' with `a4paper' for UK/EU standard size
\usepackage[letterpaper,top=2cm,bottom=3cm,left=3cm,right=3cm,marginparwidth=1.75cm]{geometry}

% Useful packages
\usepackage{amssymb, amsmath, bm, ulem, fancybox, ascmac, changepage, amsthm, amsfonts}
\usepackage[dvipdfm]{graphicx}
\usepackage[dvipdfmx]{color}
\usepackage[colorlinks=true, allcolors=blue]{hyperref}
\usepackage{tikz}
\usepackage{circuitikz}
\usepackage{pdfpages}
\usetikzlibrary{intersections,calc,arrows.meta}

\newcommand{\1}{\mbox{1}\hspace{-0.25em}\mbox{l}}

\newenvironment{subs}
  {\adjustwidth{3em}{0pt}}
  {\endadjustwidth}

\title{ユーザインターフェース プロジェクト課題レポート}
\author{計数工学科 03-230626 中根敦久}
\date{\today}


\begin{document}
\maketitle
\section{制作物}
本課題では,Figure??のようなアプリケーションを作成した.以下に設計手法および工夫点を述べる. \\
開発にはUnityを用い,Unity上でのみ実行できる形をとった.本プロダクトの最大の特徴は,Nintendo Switch(任天堂株式会社より販売される家庭用ゲーム機)で使われるリモコン(Joy-Con)を使って操作する点である.
Joy-Conは操作性に優れ(Switchの使用経験があり,操作に親しみのある人も多い),また各種センサーやボタンを搭載している.
そのため,ユーザーの潜在的な意思や直感を正確にシステムに反映しやすく,ストレスも小さいインターフェースが作成できると考えたからである.

アプリケーションの基本的な使用方法は,1.調整したいパラメータをAボタンで選択, 2.Joy-Conのスティック(円形に$360^{\circ}$動かせるボタン)を左右に動かしてパラメータを調整, 3.調整が済んだらBボタンで終了, という1.$\sim$3.の流れを繰り返していく,というものである.
実際に操作する動画は???に掲載した.

本プロダクトのデザインで主に工夫した点を,第2回講義で挙げられたいくつかの評価基準の観点から概説する.

\subsection*{操作の直感性・覚えやすさ}
上に述べた使用方法のうち,「1.パラメータの選択」の具体的な方法は,Joy-Conを対応する自分の体の部位に移動させて指定するという方式を採用した.
例えば,肩幅を調整したい時は,Joy-Conを自らの首のあたりへと移動させると,アプリケーションがモデルの型の周辺をポイントする(Figure??).この状態でAボタンを押すと肩幅を調整するモードになり,操作方法の2.へと遷移する(Figure??)
(胴回りを調整するには自らの腰のあたりに, 袖丈を調整するには腕を肘の辺りで直角に,それぞれJoy-Conを持てば良い(Figure??). 内部的には,Joy-Conの角度センサーの値を自ら作成した機械学習モデルに通して判定している).
操作するのがユーザーと同じ姿形をしている人間型のモデルであることから,このような方法がより直感的であると考えた.

また,各機能をなるべく直感的に理解できるようなデザインを心がけた.画面にはJoy-Conのボタンを現実となるべく同じ配置で描き,各ボタンの機能はその脇にできるだけ簡潔に記した.
なお,AボタンとBボタンも非言語で視覚的に理解できる記号を探したが適当なものが浮かばなかったため,英単語を採用した.


各ボタンの機能の割り当てにも注意を払った.まず,A,BボタンにはそれぞれSelect, Diselectを割り当てた.これはJoy-Conを用いる多くのゲームで一般的な,「Aボタンは選択・決定」「Bボタンは戻る」という機能に則ることで,任天堂のゲームに馴染みのあるユーザーに抵抗なく受け入れられるようにという配慮である.
また,スティックの操作方向とパラメータの増減の対応には次のような選択肢があった: 1.「左に動かすとパラメータ増加,右に動かすと減少.あるいはその逆」2.「上に動かすとパラメータ増加,下に動かすと減少.あるいはその逆」, 3. 「1.と2.をパラメータごとに変える」.
3.の案は,袖丈の調整に限っては,肩幅や胴回りと異なり縦方向の変化であるため,横方向のスティックの移動では直感に反するという懸念があったためである.
1.$\sim$3.のいずれもを実装してみたところ,個人的な体験としては「1.左右方向の移動」が最も満足のいく操作感を得られたためこれを採用した.増加方向は日常的に(数直線やゲーム「スーパーマリオブラザーズ」などで)馴染みのある「左に動かすと増加」を採用した.

さらに工夫した点として,使用不可能なボタンの説明は非表示にするのではなく薄く表示する様にした.
本アプリケーションでは複数のボタンを使いこなす必要があるため,それらの機能の説明を描画するべきであるが, これを愚直に全て描くと画面内の情報量が増えてしまい混乱の種になる.
一方で,常にその場に応じた必要最低限の説明しかなされていないければ,場面ごとに画面上の説明が変化することになりこれも認知負荷が大きいと考えた.
そこで,使用可能なボタンの機能の説明はもちろんのこと,使用不可能なボタンについても薄い色で説明を書くという方法を採用した(Figure??).
これにより,ユーザーは今使える一部の機能のみがハイライトされているため選択に迷いは生じ辛く,かつ画面の情報は色の濃淡しか変化しないため,情報が目まぐるしく変わるという事態も避けられる.
特にYボタンに関しては全く機能を割り当てていないため,常に薄い色で表示される様にした.


\subsection*{目的の達成しやすさ・成功率}
独自に加えた機能の一つに,服の透過・塗りつぶしを切り替える機能を搭載した.
ユーザーはJoy-ConのXボタンを押すことで,服が透過してモデルの全身の体型と服の寸法を比較できる状態と,服が塗りつぶされてモデルが実際に服を着ている姿を確認できる状態を瞬時に切り分けることができる.
これによりユーザーはより多くの視点から自分のデザインを評価でき,成功率の上昇(および次のトピックの主観的満足に該当する,自信を持ったデザイン)が可能になる. \\




\subsection*{主観的満足}
ユーザーのストレスをなるべく軽減することを目指した.
一つの観点として,ユーザーが「自分が今何をしているのか・どこにいるのか」を見失わないことを心がけた.
ユーザーが選択しているパラメータがインディケータで指し示されたり,モデルの一覧および今どのモデルを選択しているかが画面右上から確認できたりする機能は
この志向によるものである.インディケータにはアニメーションをつけて,ユーザーが「いまそれを選択している状態であり,編集可能である」ことが直感的にわかる様にした.









\section{ユーザーテスト}


\section{}

\end{document}